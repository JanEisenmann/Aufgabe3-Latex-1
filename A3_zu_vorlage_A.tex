\documentclass{article}
\usepackage[utf8]{inputenc}
\usepackage[T1]{fontenc}
\usepackage{natbib}
\title{Aufgabe 3 zu Muster A}
\author{Oliver Baltensperger}
 
\begin{document}
\maketitle

\section{Muster A}
Ein Mustertext ist ein an sich kurzer Text, welcher sich immer wiederholt. Der Praktische Nutzen hierbei erschließt sich z.B. für Layouter und Webdesigner, die ein fertiges Design mit irgendwelchem Text füllen müssen, um das Gesamterscheinungsbild deutlich zu machen. Ein Mustertext ist ein an sich kurzer Text, welcher sich immer wiederholt. Ein Mustertext ist ein an sich kurzer Text, welcher sich immer wiederholt. Der Praktische Nutzen hierbei erschließt sich z.B. für Layouter und Webdesigner, die ein fertiges Design mit irgendwelchem Text füllen müssen, um das
Gesamterscheinungsbild deutlich zu machen. Ein Mustertext ist ein an sich kurzer Text, welcher sich immer wiederholt.\citep{meusel2005entwicklung} \\
\\
Ein Mustertext ist ein an sich kurzer Text, welcher sich immer wiederholt. Der Praktische Nutzen hierbei erschließt sich z.B. für Layouter und Webdesigner, die ein fertiges Design mit irgendwelchem Text füllen müssen, um das Gesamterscheinungsbild deutlich zu machen.  \citep{hawking1993brief}

\begin{center}
Ein Mustertext ist ein an sich kurzer Text, welcher sich immer wiederholt. 
\end{center}

Ein Mustertext ist ein an sich kurzer Text, welcher sich immer wiederholt. Ein Mustertext ist ein an sich kurzer Text, welcher sich immer wiederholt. Ein Mustertext ist ein an sich kurzer Text, welcher sich immer wiederholt.\citep{nicolaou2013aufkommen} 

\bibliographystyle{alpha}
\bibliography{bibfile}{}

\end{document} 
