\title{CS102 \LaTeX \ \"Ubung}
\author{Oliver Baltensperger}
\date{4. November 2014}
\title{CS102 \LaTeX \ \"Ubung}
\author{Oliver Baltensperger}
\date{4. November 2014}

\documentclass{article}
 
\begin{document}

\maketitle

\section{Das ist der erste Abschnitt}
Hallo. Dies ist mein erstes \LaTeX -Dokument.
\section{Tabelle}
In folgender Tabelle sind meine bisherigen Punkte aufgelistet:
\begin{center}
\begin{tabular}{c|c|c|c}
 \ 					& Punkte erhalten 	& Punkte möglich 	& \% 	\\
\hline	\"Ubung 1 	& 10 				& 10 				& 1 	\\
		\"Ubung 2	& 10				& 10				& 1		\\
 		\"Ubung 3	& 10				& 10				& 1		\\
 		
\end{tabular}
\end{center}

\section{Formeln}
\subsection{Pythagoras}
Der Satz des Pythagoras errechnet sich wie folgt:
$ a^2+b^2=c^2 $.
Daraus k\"onnen wir die L\"ange der Hypothenuse wie folgt berechnen:
$ c=\sqrt{a^2+b^2}$
\subsection{Summen}
Wir können auch die Formel für eine Summe angeben:

\begin{center}
\begin{equation}
s=\sum\limits_{i=1}^n i=\frac{n*(n+1)}{2}
\end{equation}

\end{center}

\begin{equation}
\sin x
\end{equation}



   
\end{document} 
